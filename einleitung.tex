Viele Berechnungsprobleme sind in ihrer Essenz Suchprobleme: Optimierungsprobleme sind die Suche nach der optimalen Lösung, der Versuch eine kryptographische Verschlüsselung zu brechen ist die Suche nach dem korrekten Schlüssel. Um diese Probleme effizient lösen zu können, ergibt sich die große Relevanz der Suchalgorithmen.
Diese Arbeit ist im Rahmen des Moduls “Quantencomputer verstehen - Grundlagen und Anwendungen” an der Hochschule Düsseldorf bei Prof. Dr. Holger Schmidt entstanden. In dem Modul wurden zunächst gemeinsam die Grundlagen des Themengebiets Quantencomputer erarbeitet. Darauf aufbauend wurden durch die Studierenden Seminarvorträge vorbereitet, welche verschiedene Themengebiete tiefergehend beleuchten und im Kurs vorgestellt wurden. Dies ist die schriftliche Ausarbeitung zu dem Seminarvortrag “Suchen \& Grovers Algorithmus”. Es werden die im Modul erarbeiteten Grundlagen als bekannt vorausgesetzt.
Diese Seminararbeit fokussiert sich auf den von Dr. Lov Grover entwickelten Quantensuchalgorithmus. Außerdem werden dessen verschiedene Varianten und die möglichen Anwendungsgebiete vorgestellt.
Von zentralem Interesse ist dabei, ob Quantencomputer mit dem Grover-Suchalgorithmus zukünftig jene Probleme effizient lösen können, für die es mit klassischen Computern bisher keine geeigneten Lösungen gibt. 
Aus diesem Grund findet die Laufzeit des Suchalgorithmus besondere Betrachtung, um im Fazit diese Frage angemessen beantworten zu können. 
Als Hauptquelle diente Kapitel 6 “Suchen” aus dem Buch “Quantum Computing verstehen: Grundlagen - Anwendungen - Perspektiven” von Matthias Homeister.
TODO: Aufbau der Arbeit