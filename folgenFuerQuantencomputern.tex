Aus dem vorherigen Kapitel x ergibt sich, dass Grovers Algorithmus nicht ausreichend effizient ist, um alle NP-Probleme lösen zu können. 
Um alle Probleme aus NP lösen zu können, müssten für sehr verschiedene Probleme aus exponentiell großen Mengen von Lösungen die korrekten in polynomialer Laufzeit gefunden werden. 
Dies ist nur möglich, wenn all diese diversen Probleme eine gemeinsame Struktur aufweisen, welche auf die Fähigkeiten von Quantencomputern zugeschnitten und bisher nicht bekannt ist. 
Findet sich für ein Problem aus NP-vollständig ein effizienter Algorithmus, so ist davon auszugehen, dass ebenfalls diese gemeinsame Struktur erkannt wird und somit alle NP-vollständigen Probleme effizient lösbar werden.