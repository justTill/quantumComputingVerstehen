Nach dem der Genaue Ablauf des Algorithmus erklärt wurde, bleibt noch die Frage: Wie oft muss die Grover Iteration durchgeführt werden, um mit einer hohen Wahrscheinlichkeit das gesuchte Element $\mathbf{\hat{x}}$ zu messen?

\subsection{Geometrische Veranschaulichung}
Mithilfe der Geometrie kann die Anzahl der Iterationen bestimmt werden. Das Schritt weise erhöhen der Amplitude des gesuchten Elements kann auf der Blochs Kugel als Rotation aufgefasst werden. In der Geometrie entspricht die Spiegelung von zwei Ebenen um eine Drehung, um den Winkel $\mathbf{2 \times \beta}$ wobei $\mathbf{\beta}$ der Winkel zwischen den Ebenen ist. Dies ist in der Abbildung \ref{fig:zweiEbenen} verdeutlicht.
 \begin{figure}[hbtp]
	\centering
	\includegraphics[width=.6\textwidth]{figures/zweiEbenen.png}
	\caption{$\mathbf{R_4}$ Spiegelung an zwei Ebenen \\ Quelle: eigen Darstellung, angelehnt an HOEMEISTER}
	\label{fig:zweiEbenen}
\end{figure} 
Wie der Name schon sagt, handelt sich sich bei der Spiegelung am Mittelwert um eine Spiegelung. Die zweite Spiegelung ist das negieren von dem gesuchten Element $\mathbf{\hat{x}}$. Die Formel zu Spiegelung an einem Wert $\mathbf{m}$ lautet: $\mathbf{\alpha \rightarrow 2 \times m - \alpha}$. Ist $\mathbf{m=0}$ erhalten wir $\mathbf{\alpha \rightarrow - \alpha}$. Dies zeigt, dass die Negation eines Elementes eine Spiegelung an dem Wert $\mathbf{0}$ ist.
\\ 
\\
Der Winkel lautet $\mathbf{\sin(\beta) = }$  Skalarprodukt aus der allgemeinen Superposition und dem gesuchten Element $\mathbf{\hat{x}}$. Dies ergibt  $\mathbf{\sin(\beta) = \frac{1}{\sqrt N}}$. Daran lässt sich erkennen, das die Anzahl der Iterationen abhängig von der Anzahl der Datenbankelemente ist und nicht von $\mathbf{\hat{x}}$, welches zur Folge hätte, das wir für jedes gesuchte Element die Suche anpassen müssten.
\subsubsection{Beispielrechnung}
Nimmt man an das $\mathbf{N=4}$, dann folgt daraus, dass $\mathbf{\sin(\beta) = \frac{1}{\sqrt{4}}}$ ist. Nach Beta aufgelöst ergibt sich $\mathbf{\beta = \frac{\pi}{6}}$.
\\
Wird nun eine Grover Iteration durchgeführt, ändert sich der Winkel wie folgt: $\mathbf{\beta = \frac{\pi}{6} + 2 \times \frac{\pi}{6}  = \frac{\pi}{2}}$. Wird nun $\mathbf{sin(\frac{\pi}{2}))}$ ausgerechnet, erhalten wir $\mathbf{1}$. Dies bedeutet, das wir mit einer Drehung das gesuchte Element erreicht haben. Wird nun gemessen erhalten wir mit einer Wahrscheinlichkeit von 100 \% das gesuchte Element. Dies wird auch wie im Abschnitt  \ref{sec:spiegelnAmMittelwert}  beschrieben erwartet.

\subsubsection{Anzahl an Grover Iterationen}
Durch die rechnung konnte gezeigt werden, dass der Startwinkel $\mathbf{\frac{1}{\sqrt{N}}}$ beträgt und der Winkel nach $\mathbf{T}$ Grover Iterationen der Winkel den Wert $\mathbf{(2 \times T + 1)\times \frac{1}{\sqrt{N}}}$ hat. 
Falls für $\mathbf{T = \frac{\pi}{4}\times \sqrt{N}}$ gewählt wird, wird immer sehr nah an zu dem gesuchten Element rotiert. 
\\
In der Praktischen Umsetzung, muss $\mathbf{T}$ immer abgerundet werden, da nur eine gerade Zahl an Iterationen durchgeführt werden kann. Wird $\mathbf{T}$ aufgerundet, werden zu viele Grover-Iterationen durchgeführt und es wird sich vom gesuchten Element wieder entfernt.
\\
\\
Dadurch dass die Anzahl der Grover-Iterationen $\mathbf{ \frac{\pi}{4}\times \sqrt{N}}$ beträgt und in jeder Iteration einmal das Quantenorakel aufgerufen wird, beträgt die Laufzeit des Grover Algorithmus $\mathbf{O(\sqrt N)}$
