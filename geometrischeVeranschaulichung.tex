Nachdem der genaue Ablauf des Algorithmus erklärt wurde, bleibt die Frage: Wie oft muss die Grover Iteration durchgeführt werden, um mit einer hohen Wahrscheinlichkeit das gesuchte Element $\mathbf{\hat{x}}$ zu messen? noch offen.

\subsection{Geometrische Veranschaulichung}
Mithilfe von Regeln aus der Geometrie kann die Anzahl der Iterationen bestimmt werden.
\\
In der Geometrie entspricht die Spiegelung eines Punktes an zwei Ebenen eine Drehung um den Winkel $\mathbf{2 \times \beta}$, wobei $\mathbf{\beta}$ der Winkel zwischen den Ebenen ist. Dies ist in der Abbildung \ref{fig:zweiEbenen} verdeutlicht. Diese Eigenschaft kann sich zunutze gemacht werden. Das Schrittweise erhöhen der Amplitude des gesuchten Elements kann nämlich als Rotation aufgefasst werden. Ist der Startwinkel ($\mathbf{\beta}$) bekannt, kann ausrechnet werden, um wie viel der Grad gedreht werden muss, um das gesuchte Element zu erreichen und damit wie oft die Grover-Iteration durchgeführt werden muss.
\begin{figure}[hbtp]
	\centering
	\includegraphics[width=.6\textwidth]{figures/zweiEbenen.png}
	\caption{$\mathbf{R_4}$ Spiegelung an zwei Ebenen \\ Quelle: Anlehnung an \cite[S. 149]{Ho17}}
	\label{fig:zweiEbenen}
\end{figure} 
\noindent
\\
Wie der Name schon sagt, handelt es sich bei der Spiegelung am Mittelwert um eine Spiegelung. Die zweite Spiegelung ist das Negieren von dem gesuchten Element $\mathbf{\hat{x}}$. Die Formel zur Spiegelung an einem Wert $\mathbf{m}$ lautet: $\mathbf{\alpha \rightarrow 2 \times m - \alpha}$. Ist $\mathbf{m=0}$ erhalten wir $\mathbf{\alpha \rightarrow - \alpha}$. Dies zeigt, dass die Negation eines Elementes eine Spiegelung an dem Wert $\mathbf{0}$ ist.
\\ 
\\
Der Winkel lautet $\mathbf{\sin(\beta) = \langle s | \hat{x} \rangle}$, mit $\mathbf{s = \frac{1}{\sqrt{N}}\sum\limits_{x=0}^{N-1}|x\rangle}$, welche die Allgemeine Superposition ist. Ausrechnet ergibt dies $\mathbf{\sin(\beta) = \frac{1}{\sqrt N}}$. Daran lässt sich erkennen, dass die Anzahl der Iterationen abhängig von der Anzahl der Datenbankelemente ist und nicht von $\mathbf{\hat{x}}$. Wäre dies nicht so, hätte das zur Folge, dass für jedes gesuchte Element die Anzahl der Grover Iterationen angepasst werden müsste.
\subsubsection{Beispielrechnung: Geometrische Veranschaulichung}
Sei $\mathbf{N=4}$, dann folgt daraus, dass $\mathbf{\sin(\beta) = \frac{1}{\sqrt{4}}}$ ist. Nach Beta aufgelöst ergibt sich $\mathbf{\beta = \frac{\pi}{6}}$.
\\
Wird nun eine Grover Iteration durchgeführt, ändert sich der Winkel wie folgt: $\mathbf{\beta = \frac{\pi}{6} + 2 \times \frac{\pi}{6}  = \frac{\pi}{2}}$. $\mathbf{sin(\frac{\pi}{2})}$ ausgerechnet ergibt: $\mathbf{1}$. Dies bedeutet, dass wir mit einer Rotation (Grover-Iteration) das gesuchte Element erreicht haben. Wird nun gemessen, erhalten wir mit einer Wahrscheinlichkeit von 100 \% das gesuchte Element. Dies wird auch wie im Abschnitt  \ref{sec:spiegelnAmMittelwert}  beschrieben erwartet.

\subsubsection{Anzahl an Grover Iterationen}
Durch die Rechnung konnte gezeigt werden, dass der Startwinkel $\mathbf{\frac{1}{\sqrt{N}}}$ beträgt und der Winkel nach $\mathbf{T}$ Grover Iterationen der Winkel den Wert $\mathbf{(2 \times T + 1)\times \frac{1}{\sqrt{N}}}$ hat. 
Falls für $\mathbf{T = \frac{\pi}{4}\times \sqrt{N}}$ gewählt wird, wird immer sehr nah an das gesuchte Element rotiert. 
\\
In der praktischen Umsetzung muss $\mathbf{T}$ immer abgerundet werden, da nur eine gerade Zahl an Iterationen durchgeführt werden kann. Wird $\mathbf{T}$ aufgerundet, werden zu viele Grover-Iterationen durchgeführt und es wird sich vom gesuchten Element wieder entfernt. \cite[S. 157]{KLM07} Das Souffl\'{e} würde anfangen einzugehen
\\
\\
Dadurch, dass die Anzahl der Grover-Iterationen $\mathbf{ \frac{\pi}{4}\times \sqrt{N}}$ beträgt und in jeder Iteration einmal das Quantenorakel aufgerufen wird, beträgt die Laufzeit des Grover Algorithmus $\mathbf{O(\sqrt N)}$.


